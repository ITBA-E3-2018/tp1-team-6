%\documentclass[11pt]{article}

%\begin{document}
\chapter*{1-Rango y Resoluci\'on}

\section{Objetivo}
\indent\indent  Desarrollar un programa que calcule el Rango y Resoluci\'on de un n\'umero de punto fijo.\\
\indent El mismo recibira el signo, cantidad de bits de la parte entera y la cantidad de bits de la parte fraccionaria.
\section{Calculo de Rango y Observaciones}
\indent\indent Para poder hacer el calculo del Rango se deben tomar en cuenta los 3 parametros mencionados. Esto se debe a que se debe distinguir si el n\' umero es de tipo signado (SIGNO = 1) o no (SIGNO = 0). Esto parece ser estrictamente necesario. Sin embargo m\'as adelante veremos que no lo es.
\\ \indent Comenzamos por enunciar la definici\'on de Rango:\\
	\begin{center}
	\emph{Es la diferencia entre la magnitud representable m\'as positiva y la magnitud representable m\'as negativa}
	\end{center}

\subsection{Ejemplo de c\'alculo de Rango}
\indent\indent Introduciremos el metodo de c\'alculo mediante un ejemplo:\\
\indent Supongamos que deseamos calcular el rango \textbf{R} de un n\'umero binario \textit{signado} con $3\ bits$ de parte entera y $2\ bits$ de parte fraccionaria entonces tenemos un n\'umero con la siguiente forma:



\begin{center}
\begin{tabular}{ |c|c|c|c|c| } 
 \hline
 2 & 1 & 0 & -1 & -2 \\ 
 x & x & x  & x & x\\ 
 \hline
\end{tabular}
\end{center}

Para calcular el maximo n\'umero representable $M$ :
\begin{center}
\begin{tabular}{ |ccccc| } 
 \hline
 0 & 1 & 1 & .1 & 1 \\ 
 \hline
\end{tabular}
\end{center}
$$ 0^2 + 2^1 + 2^0 +2^{-1} + 2^{-2}  = 3.75$$
\\\\
Para calcular el minimo n\'umero representable $m$ :
\begin{center}

\begin{tabular}{ |ccccc| } 
 \hline
 1 & 0 & 0 & .0 & 0 \\ 
 \hline
\end{tabular}
\end{center}
$$ m = -2^{2} = -4$$

Por lo tanto el rango: 
$$R = 3.75 - (-4 )$$ 
$$R = 7.75 $$
\\
\indent Ahora podemos preguntarnos  C\'omo realizar el c\'alculo s\'i el n\'umero es no \textit{signado}.
\\\\
La maxima denominaci\'on vendra dada por:
\begin{center}
\begin{tabular}{ |ccccc| } 
 \hline
 1 & 1 & 1 & .1 & 1 \\ 
 \hline
\end{tabular}
\end{center}

$$M' = 2^2 + 2^1 + 2^0 + 2^{-1} + 2^{-2}$$
$$M' = 7.75$$
\\
La minima denominaci\'on vendra dada por:
\begin{center}
\begin{tabular}{ |ccccc| } 
 \hline
 0 & 0 & 0 & .0 & 0 \\ 
 \hline
\end{tabular}
\end{center}

Lo cual claramente indicia que el minimo n\'umero representable es el $0$. Por lo tanto: $$m = 0$$




Entonces: 	$$R' = M' - m'$$
		 	$$R' = 7.75 $$
		 	
		 	
Notamos que $R = R'$. Esto nos da indicios de que el rango de un n\'umero de punto fijo \textit{signado} \'o \textit{no signado} tiene n el mismo rango. A continuaci\'on demostraremos que esto es de hecho as\'i.

\subsection{Formula general para el c\'alculo del Rango}



 
%\end{document}


