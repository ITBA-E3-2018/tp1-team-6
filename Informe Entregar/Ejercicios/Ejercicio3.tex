\chapter*{3-Encoder y Demux}
\section{Encoder}
%\indent
\subsection{Objetivo}
Desarrollar un m\'odulo en Verilog el cual funcione como un encoder de 4 entradas.
\subsection{Explicaci\'on y Desarrollo}
Un encoder tiene la funcionalidad de tomar los datos de entrada y los codifica para que las salidas
representen (en n\'umero en binario) la entrada cuyo valor es 1 como se puede ver en la siguiente tabla de verdad.
\begin{center}
	\begin{table}[h!]
		\begin{center}
			\caption{Tabla de Verdad del Encoder}
			\begin{tabular}{|c|c|}
				\hline
				\textbf{INPUT} & \textbf{OUTPUT} \\
				\hline
				\textbf{a b c d} & \textbf{x y} \\
				\hline
				0 0 0 0 & z z\\
				\hline
				0 0 0 1 & 0 0 \\
				\hline
				0 0 1 0 & 0 1\\
				\hline
				0 0 1 1 & z z\\
				\hline
				0 1 0 0 & 1 0\\
				\hline
				0 1 0 1 & z z\\
				\hline
				0 1 1 0 & z z\\
				\hline
				0 1 1 1 & z z\\
				\hline
				1 0 0 0 & 1 1\\
				\hline
				1 0 0 1 & z z\\
				\hline
				1 0 1 0 & z z\\
				\hline
				1 0 1 1 & z z\\
				\hline
				1 1 0 0 & z z\\
				\hline
				1 1 0 1 & z z\\
				\hline
				1 1 1 0 & z z\\
				\hline
				1 1 1 1 & z z\\
				\hline
			\end{tabular} \\
		\end{center}
	\end{table}
\end{center}
Como las \'unicas salidas v\'alidas del encoder son aquellas cuando hay una \'unica entrada que contiene un 1, las dem\'as combinaciones de entradas se consideran como comodines para las salidas. Las cuales
se puede elegir su valor de tal forma de facilitar los cálculos para obtener la funci\'on l\'ogica.
Para poder obtener la fuci\'on l\'ogica del mismo, se hizo uso del mapa de Karnaugh el cual se muestra
a continuación.
\\
\begin{Karnaugh}
	%\caption{Tabla de Verdad del Encoder}
    \contingut{z,1,1,z,0,z,z,z,0,z,z,z,z,z,z,z}
    %\implicant{0}{2}{red}
    %\implicant{5}{15}{purple}
    %\implicantdaltbaix[3pt]{3}{10}{blue}
    %\implicantcantons[2pt]{orange}
    %\implicantcostats{4}{14}{green}
\end{Karnaugh}
\begin{Karnaugh}
	%\caption{Tabla de Verdad del Encoder}
    \contingut{z,0,1,z,0,z,z,z,1,z,z,z,z,z,z,z}
    %\implicant{0}{2}{red}
    %\implicant{5}{15}{purple}
    %\implicantdaltbaix[3pt]{3}{10}{blue}
    %\implicantcantons[2pt]{orange}
    %\implicantcostats{4}{14}{green}
\end{Karnaugh}
\\
El mapa de Karnaugh de la izquierda corresponde la salida X y el de la derecha la salida Y.
Empezando con el mapa correspondiente a la salida X, se eligieron los valores de los comodines (z) de forma conveniente para poder agrupar lo m\'as posible. Este mapa se lo resolver\'a utilizando mint\'erminos.

\begin{center}
\begin{Karnaugh}
    \contingut{1,1,1,1,0,0,1,1,0,0,1,1,0,0,1,1}
    \implicant{3}{10}{red}
    \implicant{0}{2}{blue}
\end{Karnaugh}
\end{center}

Ahora, para poder armar la funci\'on l\'ogica, se debe ver qu\'e variable (dentro de cada grupo) no cambia su valor. Y si dicha variable contiene el valor 0, entonces se lo niega ya que estamos trabajando con mint\'erminos. Se obtiene la siguiente funci\'on l\'ogica para la salida X.

\begin{center}
X = Not(c)Not(d) + a
\end{center}
Cuya tabla de verdad se muestra a continuación.
\begin{center}
	\begin{table}[h!]
		\begin{center}
			\caption{Tabla de Verdad del Encoder}
			\begin{tabular}{|c|c|}
				\hline
				\textbf{INPUT} & \textbf{OUTPUT} \\
				\hline
				\textbf{a c d} & \textbf{x} \\
				\hline
				0 0 0 & 1\\
				\hline
				0 0 1 & 1 \\
				\hline
				0 1 0 & 0\\
				\hline
				0 1 1 & 1\\
				\hline
				1 0 0 & 0\\
				\hline
				1 0 1 & 1\\
				\hline
				1 1 0 & 0\\
				\hline
				1 1 1 & 1\\
				\hline
			\end{tabular} \\
		\end{center}
	\end{table}
\end{center}

Ahora centr\'andose en el mapa correspondiente a la salida Y, se eligieron nuevamente los valores de los comodines (z) de forma conveniente para poder agrupar lo m\'as posible. Este mapa se lo resolver\'a utilizando maxt\'erminos.

\begin{center}
\begin{Karnaugh}
    \contingut{0,0,1,1,0,0,1,1,1,1,0,0,1,1,0,0}
    \implicant{0}{5}{red}
    \implicant{15}{10}{blue}
\end{Karnaugh}
\end{center}

Para poder armar la funci\'on l\'ogica, se debe ver nuevamente qu\'e variable (dentro de cada grupo) no cambia su valor. Y si dicha variable contiene el valor 1, entonces se lo niega ya que estamos trabajando con maxt\'erminos. Se obtiene la siguiente funci\'on l\'ogica para la salida Y.

\begin{center}
$$Y = a\overline{c} + c\overline{a}$$
\end{center}
Cuya tabla de verdad se muestra a continuaci\'on.
\begin{center}
	\begin{table}[h!]
		\begin{center}
			\caption{Tabla de Verdad del Encoder}
			\begin{tabular}{|c|c|}
				\hline
				\textbf{INPUT} & \textbf{OUTPUT} \\
				\hline
				\textbf{a c} & \textbf{y} \\
				\hline
				0 0 & 0\\
				\hline
				0 1 & 1 \\
				\hline
				1 0 & 1\\
				\hline
				1 1 & 0\\
				\hline
			\end{tabular} \\
		\end{center}
	\end{table}
\end{center}
Se puede notar, a trav\'es de la tabla de verdad, que $$Y = xor(a,c)$$
\section{Demux}
\subsection{Objectivo}
Desarrollaar un módulo en Verilog el cual implemente el concepto de Demux.

\subsection{Criterio e implementación}
Dado que un componenete del tipo Demux tiene por finalidad dirigir su entrada hacia una salida en particular se realizo la siguiente observación:
\begin{center}
\textit{Si la entrada es nula su salida sera nula en todas las terminales}
\end{center}
Teniendo dicho conocimiento podemos evitar realizar cualquier tipo de calculo dado que al conocer la entrada la salida queda completamente determinada.
No es así en el caso de que la entrada sea 1 dado que cualquiera de los 4 terminales de salida puede ser esocgido para reflejar el valor de entrada.
Para su implementación en Verilog se decidio implementar mediante shifteos de 1 bit. Es decir que para evitar definir la solución de manera anticipada (dado que desconocemos las consecuencias de su implementación en la vida real). En otras palabra, definir una constante puede incurrir en un gasto de energía constante mientras que la realización de un calculo cuando es necesario puede ser beneficioso.


