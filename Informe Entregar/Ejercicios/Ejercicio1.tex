%\documentclass[11pt]{article}

%\begin{document}

\chapter*{1-Rango y Resolución}

\section{Objetivo}
\indent\indent  Desarrollar un programa que calcule el Rango y Resolución de un número de punto fijo.\\
\indent El mismo recibira el signo, cantidad de bits de la parte entera y la cantidad de bits de la parte fraccionaria.
\section{Calculo de Rango y Observaciones}
\indent\indent Para poder hacer el calculo del Rango se deben tomar en cuenta los 3 parametros mencionados. Esto se debe a que se debe distinguir si el n\' umero es de tipo signado (SIGNO = 1) o no (SIGNO = 0). Esto parece ser estrictamente necesario. Sin embargo más adelante veremos que no lo es.
\\ \indent Comenzamos por enunciar la definición de Rango:\\
	\begin{center}
	\emph{Es la diferencia entre la magnitud representable más positiva y la magnitud representable más negativa}
	\end{center}

\subsection{Ejemplo de c\'alculo de Rango}
\indent\indent Introduciremos el metodo de cálculo mediante un ejemplo:\\
\indent Supongamos que deseamos calcular el rango \textbf{R} de un número binario \textit{signado} con $3\ bits$ de parte entera y $2\ bits$ de parte fraccionaria entonces tenemos un n\'umero con la siguiente forma:



\begin{center}
\begin{tabular}{ |c|c|c|c|c| } 
 \hline
 2 & 1 & 0 & -1 & -2 \\ 
 x & x & x  & x & x\\ 
 \hline
\end{tabular}
\end{center}

Para calcular el maximo n\'umero representable $M$ :
\begin{center}
\begin{tabular}{ |ccccc| } 
 \hline
 0 & 1 & 1 & .1 & 1 \\ 
 \hline
\end{tabular}
\end{center}
$$ 0^2 + 2^1 + 2^0 +2^{-1} + 2^{-2}  = 3.75$$
\\\\
Para calcular el minimo n\'umero representable $m$ :
\begin{center}

\begin{tabular}{ |ccccc| } 
 \hline
 1 & 0 & 0 & .0 & 0 \\ 
 \hline
\end{tabular}
\end{center}
$$ m = -2^{2} = -4$$

Por lo tanto el rango: 
$$R = 3.75 - (-4 )$$ 
$$R = 7.75 $$
\\
\indent Ahora podemos preguntarnos  C\'omo realizar el c\'alculo s\'i el n\'umero es no \textit{signado}.
\\\\
La maxima denominaci\'on vendra dada por:
\begin{center}
\begin{tabular}{ |ccccc| } 
 \hline
 1 & 1 & 1 & .1 & 1 \\ 
 \hline
\end{tabular}
\end{center}

$$M' = 2^2 + 2^1 + 2^0 + 2^{-1} + 2^{-2}$$
$$M' = 7.75$$
\\
La minima denominaci\'on vendra dada por:
\begin{center}
\begin{tabular}{ |ccccc| } 
 \hline
 0 & 0 & 0 & .0 & 0 \\ 
 \hline
\end{tabular}
\end{center}

Lo cual claramente indicia que el minimo n\'umero representable es el $0$. Por lo tanto: $$m = 0$$




Entonces: 	$$R' = M' - m'$$
		 	$$R' = 7.75 $$
		 	
		 	
Notamos que $R = R'$. Esto nos da indicios de que el rango de un n\'umero de punto fijo \textit{signado} \'o \textit{no signado} tiene n el mismo rango. A continuaci\'on demostraremos que esto es de hecho as\'i.

\subsection{Formula general para el c\'alculo del Rango}
\indent\indent Dado un n\'umero binario $B$ en punto fijo con $k$ bits en su parte entera y $j$ bits para su parte fraccionaria. 
 Primero trataremos a $B$ como si se tratase de un n\'umero \textit{no signado}.Para ambos casos separaremos el problema en dos partes. Por un lado nos encargaremos de calcular la parte entera y luego la fraccionaria.
\\

Parte entera:
\begin{equation}
P_{k} = 1 + 2 +  \cdots  + 2^k
\end{equation}
\begin{equation}
2P_{k} = 2 + 2^2  + \cdots + 2^k +2^{k+1} 
\end{equation}

Restandolas:
\begin{equation}
2P_{k} - P_{k} = 2^{k+1} -1 
\end{equation}
\\
%Resultado para calcular la parte entera de un numero fraccionario no signado
\begin{equation}
P_{k} = 2^{k+1}-1
\end{equation}

Para realizar el calculo de la parte fraccionaria procedemos de forma analoga:

\begin{equation}
F_{j} = \frac{1}{2} + \frac{1}{4} +  \cdots  + \left(\frac{1}{2}\right)^j
\end{equation}

\begin{equation}
\frac{1}{2} F_{j} = \frac{1}{4} + \frac{1}{8} +  \cdots  + \left(\frac{1}{2}\right)^j + \left(\frac{1}{2}\right)^{j+1}
\end{equation}
Restandolas:
\begin{equation}
F_{j} - \frac{1}{2}  F_{j} = \frac{1}{2} - \left(\frac{1}{2}\right)^{j+1}
\end{equation}
\\
De lo cual obtenemos
%Resultado para calcular la parte fraccionaria de un 
\begin{equation}
F_{j} = 1 - \left(\frac{1}{2}\right)^j
\end{equation}

Luego, como estamos tratando a $B$ como un n\'umero \textit{no signado}. 
Usando 4 y 7 :
$$R = P_{k}  + F_{j} =  \left(2^{k+1}-1 \right) + \left(1 - \left(\frac{1}{2} \right)^j\right) - 0$$
\\
Obtenemos la formula para el Rango de un n\'umero \textit{no signado}:
$$R =  2^{k+1} - \left(\frac{1}{2} \right)^j$$

Para el caso de un n\'umero \textit{signado} se tienen en cuenta $k + j - 1$ bits para calcular el n\'umero m\'as positivo representable.
Es decir:

$$P'_{k} = P_{k-1} = 2^{ (k+1)-1)} -1$$

$$P'_{k} = 2^{k} -1$$

En el caso de la parte fraccionaria es analogo al primero tratado
\begin{align*}
F'_{j} = F_{j}
F'_{j} = 1 - \left(\frac{1}{2}\right)^j
\end{align*}



Para obtener la minima representaci\'on posible:
\begin{equation}
m''_{k} = -2^{k}
\end{equation}

Por lo tanto ya tenemos todo lo necesario para calcular $R'$ el rango de un n\'umero signado:

\begin{align*}
R' &= P'_{k} + F'_{j} - m''_{k} \\
R' &= \left(2^{k} -1\right) + \left(1 - \left(\frac{1}{2}\right)^j\right) - \left(-2^{k}\right)
\end{align*}
 Reordenando y agrupando convenientemente
 $$R' = 2^{k+1} - \left(\frac{1}{2}\right)^j$$
 
Finalmente hemos demostrado que el Rango de un n\'umero en punto fijo no depende de si el n\'umero es \textit{signado} o \textit{no signado}
Esto presenta una gran ventaja a nivel computacional ya que no son necesarias consideraciones extra y simplifica el c\'alculo a una sola expresi\'on

\subsection{Formula general para el c\'alculo de la Resoluci\'on}
Un n\'umero en punto fijo permite expresar n\'umeros reales con cierto grado de precisi\'on. Esta caracteristica esta limitada por la cantidad de bits que se dediquen a la parte fraccionaria.
\\
\indent La resoluci\'on vendra dada por:
\begin{equation}
Resoluci\acute{o}n = \left(\frac{1}{2}\right)^j
\end{equation}

Y es indistinto si el n\'umero es \textit{signado} o es \textit{no signado}

 
 
 
%\end{document}


